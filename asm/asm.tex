\documentclass[a4paper,12pt]{article}
\usepackage{listings}

\begin{document}

\title{Assembly Language Notes}
\author{Kirill Kozlov}
\date{\today}
\maketitle

\section{x86 Assembly Guide}
Some text ... 
\subsection{Registers}
\paragraph{}
Modern (i.e. 386 and beyond) x86 processors have eight 32-bit general purpose registers.

\subsection{Memory and Addressing Modes}
fdsfds

\subsection{Instructions}

\subsection{Calling Convention}

\begin{lstlisting}[language=C]
int	i = 0;
\end{lstlisting}

\newpage
\section{x86\_64 Assembly Guide}
fdsjfds
\subsection{Registers}
\subsection{Memory and Addressing}
\subsection{Instructions}
\subsection{Calling Convention}
\begin{itemize}
\item Registers \%rbp, \%rbx and \%r12 through \%r15 belong to the calling function and the called function is required to preserve their values.
\item The conventional use of \%rbp as a frame pointer for the stack frame may be avoided by using \%rsp to index into the stack frame. This technique saves two instructions in the prologue and epilogue and makes one additional general-purpose register (\%rbp) available.
\item Argument classification
\begin{itemize}
\item \textsc{pointer}
\item \textsc{integer}
\item \textsc{SSE}
\item \textsc{SSEUP}
\item \textsc{x87, x87UP}
\end{itemize}
\end{itemize}
\newpage
\pagenumbering{roman}
\tableofcontents
\pagenumbering{arabic}


\end{document}
