\documentclass[a4paper,12pt]{article}
\usepackage{listings}

\begin{document}

\title{Assembly Language Notes}
\author{Kirill Kozlov}
\date{\today}
\maketitle

\section{x86 Assembly Guide}
Some text ... 
\subsection{Registers}
\paragraph{}
Modern (i.e. 386 and beyond) x86 processors have eight 32-bit general purpose registers.

\subsection{Memory and Addressing Modes}
fdsfds

\subsection{Instructions}

\subsection{Calling Convention}

\begin{lstlisting}[language=C]
int	i = 0;
\end{lstlisting}

\newpage
\section{x86\_64 Assembly Guide}
fdsjfds
\subsection{Registers}
\subsection{Memory and Addressing}
\subsection{Instructions}
\subsection{Calling Convention}
\begin{itemize}
\item Registers \%rbp, \%rbx and \%r12 through \%r15 belong to the calling function and the called function is required to preserve their values.
\item The conventional use of \%rbp as a frame pointer for the stack frame may be avoided by using \%rsp to index into the stack frame. This technique saves two instructions in the prologue and epilogue and makes one additional general-purpose register (\%rbp) available.
\item Argument classification
\begin{itemize}
\item \textsc{pointer}
\item \textsc{integer}
\item \textsc{SSE}
\item \textsc{SSEUP}
\item \textsc{x87, x87UP}
\end{itemize}
\end{itemize}

\section{AT\&T vs Intel Syntax}
\begin{tabular}{|c|c|}
\hline
\lstset{language=[x86masm]Assembler}
as & nasm
\\
\hline
\begin{lstlisting}
.text
movq	%rax, %rcx
movq	$0x2a, %rax
movq	foo(%rip), %rax	# content
leaq	foo(%rip), %rax	# address
movabs	$foo, %rax 	# address
.data
foo:	.quad 0
\end{lstlisting}
&
\begin{lstlisting}
section .text
mov	rcx, rax
mov	rax, 2ah
mov	rax, [rel foo]	; content
mov	rax, foo 	; address
mov	rax, [abs foo]	; address
section .data
foo	dw 42
bar	db 0
\end{lstlisting}
\\
\hline
\end{tabular}

\newpage
\section{Code Snippets}
\lstset{language=[x86masm]Assembler}
\textsc{if statement}
\begin{lstlisting}[frame=single]
if:
	<condition code>
        jcc	else
then:
	<then logic code>
        jmp	end
else:
	<else logic code>
end:
\end{lstlisting}
\textsc{while loop}
\begin{lstlisting}[frame=single]
	<code before the loop>
	mov rcx, 2ah
loop1:
	<code to loop through>
        loop loop1
        <code to execute after loop>
\end{lstlisting}
\textsc{for loop}
\begin{lstlisting}[frame=single]
for:
	<condition code>
	jcc forcode
        jmp end
forcode:
	<for loop code>
        jmp for
end:
\end{lstlisting}

\newpage
\pagenumbering{roman}
\tableofcontents
\pagenumbering{arabic}


\end{document}
